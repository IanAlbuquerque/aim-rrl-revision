\subsection{Correctness of Pick-Up-Sticks Algorithm}
\label{corr_pus}

For the sake of completeness, we include:
\begin{theorem} [part of Theorem 3.1 of \cite{ACJKLW07}]
\label{theorem_pick_up_sticks}
A black and white pattern $P$ is a strip-rule pattern if and only if the Pick-Up-Sticks algorithm results in a grid with  all of its cells gray.
In that case, the reverse list of picked up rows
and columns is a SRRL for $P$.
\end{theorem}

Since Theorem 3.1 of \cite{ACJKLW07} is not proven in their conference version,
we include a proof for the sake of completeness.

\begin{proof}
Let's separate the proof into the two following parts:
\begin{itemize}
\item ($P$ is a strip-rule pattern) $\rightarrow$ (The Pick-Up-Sticks algorithm results in an all-gray grid)

Suppose $P$ is a pattern that can be generated by a SRRL
$S = (s_{1},s_{2},\cdots,s_{m})$ of size $m$.  Suppose we have picked up some
sticks.
Let $s_{i}$ is the highest-index rule from $S$ that still has a
row/column $c$ such that $c$ is not all-gray.
If there is no such $s_{i}$, then we must have picked up everything.
For every integer $j$, $i < j \leq m$,
all the rows/columns of $s_{j}$ must have already been entirely colored gray.
That means that all cells of $c$ that belong to a rule $s_{j}$, with
$i < j \leq m$, are gray. The cells of $c$ that are not gray must have
the same color as the rule $s_{i}$, as they will not be overwritten
by any rule $s_{j}$ with $i < j \leq m$.
Hence, $c$ must be pseudo-monochromatic and then the Pick-Up-Sticks may proceed. As a result, the algorithm never gets stuck until the all-gray grid is reached.

\item (The Pick-Up-Sticks algorithm results in an all-gray grid) $\rightarrow$ ($P$ is a strip-rule pattern)

The reverse list of the picked up rows and columns is a SRRL since on every iteration of the algorithm we only pick up strips of $P$. If the Pick-Up-Sticks algorithm results in an all-gray grid, every row and column of $P$ has been picked up by the end of the algorithm. Hence, the reversed list of picked up rows and columns is a SRRL that generates $P$, making $P$ a strip-rule pattern.
\end{itemize}

This ends our proof.
\qed
\end{proof}

The same proof works (the number of rules may differ by one) if
the Pick-Up-Sticks algorithm results in a grid with none of its cells black.
