\newcommand{\hs}{\hat{s}}
\newcommand{\hS}{\hat{S}}
\newcommand{\hU}{\hat{U}}

\section{Algorithm}
The flow of our algorithm closely follows that of the $O(n^3)$ algorithm given by "Applegate et al.".
We similarly create a segment graph of the reachable states in a MPUS of the pattern.
However, we create this graph faster by grouping similar segments into S-groups, which can be processed together.

\subsection{Setup}
We begin by reading the input pattern into an equivalence class list and creating the first node in our segment graph.

\subsubsection{Equivalence Class List.}
 We are given as input an $n_R \times n_C$ grid of black and white cells.
Assign each of the rows an index from 1 to $n_R$ from top to bottom, similarly each column, an index from 1 to $n_C$ from left to right.
 Group rows and columns into equivalence classes and order the classes by number of black cells as previously described in the Subsection \ref{ss_equiv}.
Precisely, construct the hierarchical array:
$$C_{x_{0}}, R_{y_{N_{R}}}, C_{x_{1}}, R_{y_{(N_{R}-1)}}, \cdots, R_{y_{1}}, C_{x_{(N_{C}-1)}}, R_{y_{0}}, C_{x_{N_{C}}}$$
(this array may start and/or end with row equivalence classes, instead of columns as above).

We will need to determine from a range of column indexes which columns are
also in a range of equivalence classes, and the similar question with
rows instead of columns.
A data structure such as an orthogonal range tree \cite{KOS2000}
accomplishes these queries in $O($log $n)$ time with $O(n \log n)$
space and setup time.

\subsubsection{S-group.}
 %All of this is definition of segment and doesn't need to be here.
% We know that an optimal solution exists in the form of an MPUS.
% Any RRL created by MPUS can be divided into a sequence of segments, each segment containing all the rules applied when a given pair of equivalence classes were active.
% A segment can be uniquely identified by the 3-tuple label $\{E_1,E_2,U'\}$, where $E_1$ and $E_2$ are the active classes at the time of the segment, and $U$ is the set of members of $E_2$ which have not yet been picked up.
% (No members of $E_1$ will have been picked up because it has just become available for pick-up.) %Note this language has been heavily lifted from "Applegate et al.".
 Define an S-group $(E_1, E_2)$
(for equivalence classes $E_1$ and $E_2$, where the order of $E_1$ and $E_2$
in the tuple above matters)
 to be a collection of all segments $\{C_1,C_2,U'\}$
such that $C_1 = E_1$ and $C_2 = E_2$.
 To completely represent each of these segments, an S-group $(E_1, E_2)$
maintains a list of segments $S$ and a master list $U$.
 This list $U$ will be used to keep track of each of the segment's individual
 list $u$, as described below.
 This allows each segment to be stored in constant space, while the S-group gets stored in $O(|E_1|+|E_2|)$ space.

 The list $U$ contains the index of every member of $E_2$ ordered such that for every segment $\{E_1,E_2,U'\}$ in the S-group, the first $|U'|$ members of $U$ comprise the same set as $U'$.
Such an ordering is guaranteed to exist by the Containment Lemma from
 Subsection \ref{ss_cont}.
We generate this ordering by noting that indexes appearing in more segments appear earlier in the list $U$.

$S$, the list of segments maintained by the S-group, keeps these
segments ordered by the size of their set $U'$.
Each segment $s_i$ will have three values, $u(s_i)$, $d(s_i)$, and $p(s_i)$.
$u = |U'|$.
%We can ignore any segment with $u=0$ by OPTIMAL SEGMENT LEMMA, so we will assume $u>0$.
$d$ gives the minimum number of sticks required to reach the state of segment starting from the original pattern.
We want to maintain the shortest path to each node, so we use the value $p$ to store a reference to the segment right before this one on this path. %WOrding. I dont know how to say this better.

%Finally we create an $n \times n$ array to store our S-groups.
\subsubsection{Origin S-group.}
We begin with the S-group $(E_1,E_2)$ with $U = \overline{E_2}$ and $S$ comprised of a single segment $(u,d,p) = (|\overline{E_2}|,0,null)$, where  $\overline{E_2}$ represents the complete set of the members of $E_2$.
The classes $E_1$ and $E_2$ are the first and last members of the
hierarchical array.
We will explore outwards from this S-group in a breadth first fashion, enumerating all reachable segments.
The graph on the S-groups is guaranteed to be acyclic because the number of picked up classes is strictly increasing along each edge.




\subsection{Finding the Next S-group}


Suppose we have some S-group $(E_1, E_2)$ with associated lists $U$ and $S$.
Without loss of generality we will assume $E_1$ to be a column class.
We wish to generate all S-groups reachable from this S-group.
The two S-groups immediately reachable from this S-group correspond to completely removing either class $E_1$ or $E_2$.

To determine the cost of removing a class $E_i$ from a segment $s_j$,
we count how many members of $E_{3-i}$
can be picked up for free, as described in detail later.
If two members of $E_i$ may be picked up with the same stick
while $E_1$ and $E_2$ are the active classes, we will call the
two members {\em contiguous}. It follows from Proposition
\ref{corollary_hierarchical_sequence}
%and the discussion at the end of Subsection \ref{ss_ordering}
that a pair of columns is contiguous in an S-group $(E_1,E_2)$ if every column
between them is either already picked up or of type $E_1$ or $E_2$.

Before the class $E_i$ can be removed from a segment  $(E_1,E_2,U')$
(all the members of the class being picked up),
we pick up every embedded member of $E_{3-i}$.
To determine the cost of moving to the next segment,
we must count how many of the embedded columns are also contiguous.
Then any embedded columns must be removed from either
the set $U'$ (if $E_1$ is removed) or $E_1$ (if $U'$ is removed) in the next
segment and, by doing this for all segments, in the next S-group.
In fact, we process one S-group at a time, and obtain segments of
another at most two S-groups, as explained below.

Because $E_1$ and $E_2$ are not symmetric ($E_2$ comes with its set $U$), we will describe the process of removing classes $E_1$ and $E_2$ separately.

%Possibly move some more things here out of the E1 and E2 subsections

\subsubsection{Remove $E_1$.}
Let $E_3$ be the equivalence class adjacent to $E_1$ in the equivalence class list, which has not yet been removed.
The S-group corresponding to the result of removing $E_1$ will be $(E_3,E_2)$
with lists $\hU$ and $\hS$.

\paragraph{Count Contiguous Consecutive Pairs of Columns of $E_1$.}
To determine the distances to the segments of the next S-group,
we must count the number of contiguous $E_1$ ranges.

To do so, we must determine the range of equivalence classes which have not yet been picked up.
Let $E_1$ be the column class $C_a$.
If $E_2$ is also a column class call it $C_b$.
If $E_2$ is a row class, then $C_b$ is the column class adjacent to $E_2$ in the equivalence class list that has already been picked up.
%Note that this assumes padding the list with imaginary classes C_0 and C_m+1
We assume $a<b$, with the other case being symmetric.
By Proposition \ref{corollary_hierarchical_sequence},
 a column class $C_i$ has not been picked up if and only if $i\in (a,b)$.
%Notation here. a could be bigger than b, but this seems pretty clear to me.


For each pair of adjacent members of $E_1$ we check to see if
they are contiguous.
Precisely, for each member $e_i \in E_1$, we query our column range tree for members in the range $[a+1,b-1]$ with index in the range $[e_i,e_{i+1}]$,
where $e_{i+1}$ is the column of $E_i$ that has the smallest index
among those with index higher than $e_i$ ($e_{i+1}$ is the next column of $E_i$
after $e_i$).
Let $c$ be the number of pairs of consecutive members of $E_1$ which are also contiguous.
The value $(|E_1| - c)$ corresponds to the number of sticks required to completely remove $E_1$ after any embedded columns of $E_2$ are picked up.

\paragraph{Get Embedded $E_2$ Blocks.}
If $E_2$ is also a column class, we must count and remove any members of $E_2$ embedded in members of $E_1$.
Similar to the way we checked that columns of $E_1$ were contiguous, we will check if adjacent columns of $E_1$ are both contiguous and contain columns of $E_2$.
In order to accurately keep track of the cost to remove these embedded columns we must give each embedded column a tag based on which two columns surrounded the embedded column.
Two members of $E_2$ can be picked up with the same stick if and only if they have the same tag.

When an embedded column is found, tag it with the index of the $E_1$ column to its left.
Then add this column to a list, $B$, sorted by the column's index.
After we have found all of the embedded columns, we are ready to generate $\hU$ and $\hS$.

\paragraph{Build the Next S-group.}
Begin with $\hU$ as an empty list.
We also need a set $T$ to keep track of which tags have been accounted for.
Iterate through $U$, searching for each member $m$ of $U$
 in the set of embedded columns, $B$.
If $m\not \in B$, append $m$ to the end of $\hU$.
Otherwise add $m$'s tag value to the set $T$ if it is not already there. %Stronger way to say this?
Once we have checked all of the members $U'$  in a segment $s_i$
($U'$ being the first $u(s_i)$ elements of $U$) ,
add a new segment $\hs_i$ to $\hS$ with the following values.
$$u(\hs_i) = |\hU|$$
$$d(\hs_i)  = d(s_i) + |T| + |E_1| - c$$
$$p(\hs_i)  = s_i$$
(note that $|\hU|$ and $|T|$ are computed for the sets  $\hU$ and $T$ exactly
when finish processing the last element of $U'$ while iterating through $U$;
$\hU$ and $T$ can change later on).
If $\hS$ already contains a segment $\hs_j$ with $u(\hs_j) = u(\hs_i)$,
then keep only the segment with shorter distance $d$ in the set $\hS$,
 and remove the other one. %wording?
As an aside, one can see that if we don't remove duplicates,
if two segments from the same S-group $s_i$ and $s_j$ have
$u(s_i) > u(s_j)$,  then
$u(\hs_i) > u(\hs_j)$,  which is used in the proof of the
Containment Lemma (Subsection \ref{ss_cont} in the Appendix).

\subsubsection{Remove $E_2$.}
Let $E_3$ be the equivalence class adjacent to $E_2$ in the equivalence class list, which has not yet been removed.
The S-group corresponding to the result of removing $E_2$ will be $(E_3,E_1)$ with lists $\hU$ (whose elements are members of the class $E_1$) and $\hS$.


\paragraph{Count Contiguous Consecutive Pairs of Columns of $E_2$.}
To count contiguous ranges of $E_2$, we must count the contiguous ranges within each segment separately.  We use a counter $c$, initially set to $0$.
Fortunately, if a pair of columns is contiguous in one segment, it is also contiguous in all larger (with bigger value of $u$) segments.
For each index $u_j \in U$ (elements of $E$ are indices of columns),
 insert $u_j$ into a sorted list of indexes, $I$.
Get the predecessor and successor of $u_j$ in $I$ and check if these ranges from $u_j$ to its neighbors are contiguous.
If either of these ranges exist and are contiguous
 then this column can be picked up for free.
If not, we increment our cost counter $c$.
Once we have iterated over the first $u(s_i)$ blocks, we save the state of our counter in a value $c_i = c$.
This value corresponds to the cost to completely pick up the columns in the segment $s_i$ after any embedded members of $E_1$ have been picked up.
Proceed (with $c$ possibly increasing) until we finish the list $U$.


\paragraph{Get Embedded $E_1$ Blocks.}
If $E_2$ is also a column class, we must count and remove the embedded columns of $E_1$.
Each segment could have a unique number of embedded columns, where the larger the segment, the more columns can be embedded and the smaller the resulting segment will be in the next S-group.
As an aside, this is an argument used in the proof of the Containment Lemma
from \ref{ss_cont}.
To generate the ordering of $\hU$ and the values in $\hS$ and
we must carefully count the embedded columns.

We iterate through the columns of $U$ keeping track of which columns
 are embedded and will get picked up.
To help us with this task, we start with a sorted list $V$
of the indexes of $E_1$, an empty list for $\hU$,
 and a counter $d$, initialized to $d=0$,
 to measure the number of required sticks.
Also start with $B$, a set of columns of $E_1$, initialized as the empty set.

Similar to the way we counted contiguous blocks, for each column $u_i \in U$ we insert $u_i$ into a sorted list of indexes $I$.
Get the predecessor and successor of $u_i$ in $I$, called $u_j$ and $u_k$
respectively, if they exist (in which case $j < i$ and $k < i$).
%To avoid reading embedded columns more than once, we give each $u_i$ a right mark or a left mark if embedded columns to its right or left respectively have been picked up and accounted for.
%If the successor does not have a right mark, then we check if the range is contiguous.
%If so, we query for the list of columns embedded between $u_i$ and its successor.
%Do the same for the predecessor, checking for a left mark.
Use range search to check if $u_j$ and $u_i$ are contiguous,
and if $u_i$ and $u_k$ are contiguous; if a pair does not exist,
treat it as not being contiguous.
If neither of these two pairs is contiguous, do nothing.
If exactly one of these pairs is contiguous, use binary search in $V$
to obtain the set of columns of $E_1$ embedded between the pair,
add this set to $B$, and increment $d$.
If both of these pairs are contiguous, use binary search to determine
if there are elements of $\overline{E_1}$
(defined earlier as all the columns of class $E_1$)
 between $u_j$ and $u_i$ and between
$u_i$ and $u_k$; in which case we increment $d$.
%If any embedded columns were found increment $d$ and add the set of newly found embedded columns to set $B$ (which was initialized as the empty set).
After we have iterated through $u(s_i)$ columns of $U$, it is time to add to
$\hU$ and create a new segment $\hs_i \in \hS$ with the following values.
$$u(\hs_i) = |V| - |B|$$
$$d(\hs_i) = d(s_i) + d + c_i$$
$$p(\hs_i) = s_i$$ %Is it possible to end up with two segments of the same distance again? I dont want to think about it right now. @@
Remove all of the members of $B$ from $V$,
 then add all of these members to the beginning of $\hU$.
Reset $B$ to be the empty set.
Continue iterating through $U$.

Once we have finished iterating through $U$, add the remaining elements in $V$ to the start of $\hU$.

\paragraph{Build the Next S-group.}
If $E_2$ is a row class, then $\hU = \overline{E_1}$
 and $\hS$ contains one segment $\hs$.
To find $\hs$, we iterate over each segment $s_i \in S$
looking for the segment $s_i$ with minimum value $d(s_i) + c_i$.
$\hs$ has the following values:
$$u(\hs) = |\overline{E_1}|$$
$$d(\hs) = d(s_i) + c_i$$ %c_i doesnt make any sense here. c needs to be incorporated into the segments. ://///
$$p(\hs) = s_i$$ %you know, i still dont like the way i'm doing this. but I dont really mind breaking up the text. so i dont know.
%Maybe we incorporate the c_i thing into the segment similar to the rest.

If $E_2$ is a column class, $\hU$ and $\hS$
 were created while we found embedded columns.
%I could rewrite this so get embedded blocks actually returns the sets of embedded classes and then use it to compute next S-group in order to match the structure. This way seems easier if possibly harder or trickier.



\subsection{Merging Identical S-groups}

Once we have generated a new S-group $(E_1,E_2)$ we add it to a two dimensional table, where the row is determined by $E_1$'s index in the original list of equivalence classes, and the column similarly determined from $E_2$'s index.

If an S-group $(E_1,E_2)$ already exists, we must merge the two S-groups.
%At most three merges will occur for any given S-group type. Is this necessary here? This is more about the runtime.
Given two $(E_1, E_2)$ S-groups, $G_1 = \{U_1,S_1\}$ and $G_2 = \{U_2,S_2\}$, we will compute a new S-group, $G_3 = \{U_3,S_3\}$, that encompasses both of these S-groups which will then get stored in our table.
We iterate through both $G_1$ and $G_2$ concurrently in order to create $G_3$,
as described below. One starts with $U_3$ and $S_3$ being empty.


Merge $S_1$ and $S_2$, maintaining the ordering based on $u$
Iterate through each $s_i$ in this merged list.
Let $S_j$ be the list which contains $s_i$.
Remove elements from the start of $U_j$,
adding them to the end of $U_3$ until $|U_3| = u(s_i)$.
(Do not add a value to $U_3$ if it is already in $U_3$.)
This works because of the Containment Lemma from the
 Appendix \ref{ss_cont}. Actually, the proof in \cite{ACJKLW07}
of the Containment Lemma relies on proving the fact that
 all the values of $u(s)$ with $s \in S_1$ are at most the minimum
of the values of $u(s)$ with $s \in S_2$, or vice versa.
Add $s_i$ to $S_3$.
If $G_1$ and $G_2$ both have segments such that $u(s_a) = u(s_b)$, choose the segment of smaller distance to add to $S_3$.

\subsection{Finding the optimum SRRL}

While we build the graph, we keep track of the segment which is a valid endpoint of the smallest distance.
We define a valid endpoint to be a segment which is completely gray
(this is since we solve the modified version of the problem;
for the original problem, we would have a segment which is completely
white and gray).
A segment $(E_1,E_2)$, is an endpoint if $E_1=E_2$.
(in the original version, we also have the case where
$E_1$ and $E_2$ are adjacent in the equivalence class list and
 the cell where $E_1$ and $E_2$ intersect is white in the original pattern).
Then once the graph is finished, we build a path to the valid endpoint using the values stored in $p$.
This path will give the optimal order to remove equivalence classes, which can then be translated into the optimal list of rectangle rules.

This algorithms returns the optimum solution, as follows from all the
discussion above.
