\section{Time and Space Complexity}
\label{s_run}


In this section we will show the time and space complexity of the previously defined algorithm is $O(nN$ log $n)$ and $O(nN)$ respectively -- where $N$ is the number of equivalence classes and $n$ is the number of rows and columns in the original pattern. 
The ideas of our proof are partially taken from 
a more complete version of \cite{ACJKLW07},
 which showed the number of reachable segments to be in $O(n^2)$.
Indeed, our contribution is a faster way of processing a segment,
cutting down this processing time down from $O(n)$ to $O(\log n)$.

Following \cite{ACJKLW07},
we will show that if the complexity for an S-group $(E_a,E_b)$ is in $O(|E_a| + |E_b|)$, then the overall complexity of all S-groups is $O(nN)$.
For each S-group, add the first term of its complexity, $O(|E_a|)$, to one two dimensional array, and its second, $O(|E_b|)$, to a second array -- each at location $(E_a,E_b)$.

\[
\begin{bmatrix}
    E_1 &  E_1 & \dots & E_1\\
    E_2 &  E_2 & \dots & E_2\\
    \vdots & \vdots & \ddots & \vdots \\
    E_N &  E_N & \dots & E_N\\
\end{bmatrix}
\begin{bmatrix}
    E_1 &  E_2 & \dots & E_N\\
    E_1 &  E_2 & \dots & E_N\\
    \vdots & \vdots & \ddots & \vdots \\
    E_1 &  E_2 & \dots & E_N\\
\end{bmatrix}
\]
 
By noting that the sum of all members of all equivalence classes equals the total number of rows and columns, we get $\sum_{i=0}^N{|E_i|} = n$. 
The sum of the columns in the first matrix and the sum of the rows in the second matrix both equal $n$. 
The sum of all terms in both matrices is $2nN$, so the complexity of all S-groups is in $O(nN)$.

The space complexity of the algorithm is determined by the sizes of the lists storing the S-groups. 
Each S-group maintains a list $U$, which holds members of $E_b$,
 and is therefore in $O(|E_b|)$. The list $S$ contains all segments. 
Each segment is guaranteed to have a unique size $u$, 
and the values of $u$ are positive values less than $|E_b|$. 
The complexity of each segment is constant, so we again have $O(|E_b|)$. 
As we have previously shown, since each segment is in 
$O(|E_a| + |E_b|)$, the overall complexity is $O(nN)$. 

Each operation we perform on an S-group $(E_a,E_b)$ happens in either 
$O(|E_a|$ log $n)$ or $O(|E_b|$ log $n)$, as indeed 
every ``check" from the algorithm's description takes time $O(\log n)$,
after $O(n \log n$ initialization of the range search data structure
or the ordered list represented by a balanced binary tree.
So the runtime for processing the equivalence class list is $O(nN$ log $n)$.

Since reading the input takes $O(n^2)$ time and $N$ is in $O(n)$
(this follows immediately from the Monotonicity Property,
 Theorem \ref{theorem_strip_rule_patterns} in the Appendix),
 we relax our bounds and say our space complexity is $O(n^2)$
 and the overall runtime is $O(n^2$ log $n)$.
