\subsection{Segments Properties}
\label{ss_seg}

This section contains the proofs of the properties discussed earlier regarding segments.

\subsubsection{Segment Redefinition}

Earlier, we mentioned that the definition given by "Applegate et al."'s paper
\cite{ACJKLW07} is redundant. They define a segment as being a tuple of five elements $(E_{x},c_{x},E_{y},c_{y},U)$, where $E_{x}$ and $E_{y}$ are the active equivalence classes, $c_{x}$ and $c_{y}$ are the colors of the members $E_{x}$ and $E_{y}$ (that are pseudo-monochromatic), and $U$ is the subset of members left unpicked of $E_{y}$ ($U \subset E_{y}$). The next statement shows that the colors are redundant, allowing us to reduce the definition of a segment to $(E_{x},E_{y},,U)$.

\begin{proposition}
\label{corollary_known_colors}
Let $E_{1}$, $E_{2}$, $\cdots$, $E_{N}$ be the hierarchical array of a pattern. Given $a$ and $b$ such that $1 \leq a < b \leq N$ it is possible to know what colors correspond to each active class of a label ($E_{a}$,$E_{b}$).
\end{proposition}

\begin{proof}
Given $a$ and $b$ such that $1 \leq a < b \leq N$.

\begin{itemize}
\item If $E_{1} \in C$ and $E_{N} \in C$, we have that: If $E_{a} \in C$, $E_{a}$ is white. If $E_{a} \in R$, $E_{a}$ is black. If $E_{b} \in C$, $E_{b}$ is black. If $E_{b} \in R$, $E_{a}$ is white.

\item If $E_{1} \in R$ and $E_{N} \in R$, we have that: If $E_{a} \in C$, $E_{a}$ is black. If $E_{a} \in R$, $E_{a}$ is white. If $E_{b} \in C$, $E_{b}$ is white. If $E_{b} \in R$, $E_{a}$ is black.

\item If $E_{1}$ and $E_{N}$ are from different types and $E_{1}$ and $E_{N}$ are initially black, we have that: If $E_{a} \in C$, $E_{a}$ is black. If $E_{a} \in R$, $E_{a}$ is white. If $E_{b} \in C$, $E_{b}$ is white. If $E_{b} \in R$, $E_{a}$ is black.

\item If $E_{1}$ and $E_{N}$ are from different types and $E_{1}$ and $E_{N}$ are initially white, we have that: If $E_{a} \in C$, $E_{a}$ is white. If $E_{a} \in R$, $E_{a}$ is black. If $E_{b} \in C$, $E_{b}$ is black. If $E_{b} \in R$, $E_{a}$ is white.
\end{itemize}

This happens because when we pick from either sides, the color alternates for each item picked up as well as its type. Hence, the color of $E_{a}$ will always match the color of $E_{1}$ if they have the same type and differ if they are from different types.
The same will happen with $E_{b}$ and $E_{N}$.
\qed
\end{proof}

It is a consequence of this corollary the fact that having a segment $S=(E_{k_{1}},c_{k_{1}},E_{k_{2}},c_{k_{2}},U)$, $U \subset E_{k_{2}}$, is redundant. We may start referring to a segment as only $S=(E_{k_{1}},E_{k_{2}},U)$, $U \subset E_{k_{2}}$.

\subsubsection{}
\begin{proposition} [statement 2(a) of Lemma 4.2 of \cite{ACJKLW07}]
\label{p_embed}
%Suppose we have two sets $U_x$ and $U_y$
%of active blocks of the two active classes $E_x$ and $E_y$.
%Assume after picking up some sticks,
%$E_z$ suplants $E_x$  as an active class, while $E_y$ remains active,
%and this is done picking up the minimum number of sticks.
%Then one optimum solution is to pick up all the blocks of $E_y$
%embedded into $E_x$, followed by picking up the resulting
%maximal sticks of $E_x$.
Suppose we have two sets $U_x$ and $U_y$
of active blocks of the two active classes $E_x$ and $E_y$.
Suppose there is an optimal solution that completely picks up $E_x$ before
completely picking up $E_y$.
Then there also exists an optimal solution that picks up all embedded
members of $E_y$ before picking up $E_x$.
\end{proposition}
Since Lemma 4.2 of \cite{ACJKLW07} is not proven in their conference version,
we include a proof for the sake of completeness.
\begin{proof}
This follows closely from the definition of embedded members, which can be
effectively picked up at zero net cost along with the blocks embedding them.
This is easy to observe in an example: consider the leftmost three columns
of Fig.~\ref{fig:embedment_example}.
Here we see a member of $C_0$ (the white column) embedded in two blocks of
$C_2$.
We could pick up the embedding blocks using two rules, or we could first
pick up the embedded block followed by picking up both embedding blocks
using only one wider rule.
The overall result is still two rules which pick up all members of the
embedding class.
This is, without loss of generallity, either a helpful or irrelevant action.
So in our solution we will always choose to remove the embedded blocks
before the embedding blocks.
\qed
\end{proof}


\subsubsection{Containment Lemma} [Lemma 4.3 of \cite{ACJKLW07}]
\label{ss_cont}
Let $S$ and $S'$ be two segments on the same equivalence classes, then either $U \subset U'$ or $U' \subset U$.

Intuitively,
this follows from the fact that we have a strict ordering on the equivalence classes that could embed columns in $U$. See "Applegate et al."'s paper
\cite{ACJKLW07} for a rigorous inductive proof of this lemma by the same name.
